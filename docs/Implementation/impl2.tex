\documentclass[]{article}
\usepackage[T1]{fontenc}
\usepackage{lmodern}
\usepackage{amssymb,amsmath}
\usepackage{ifxetex,ifluatex}
\usepackage{fixltx2e} % provides \textsubscript
% use upquote if available, for straight quotes in verbatim environments
\IfFileExists{upquote.sty}{\usepackage{upquote}}{}
\ifnum 0\ifxetex 1\fi\ifluatex 1\fi=0 % if pdftex
  \usepackage[utf8]{inputenc}
\else % if luatex or xelatex
  \ifxetex
    \usepackage{mathspec}
    \usepackage{xltxtra,xunicode}
  \else
    \usepackage{fontspec}
  \fi
  \defaultfontfeatures{Mapping=tex-text,Scale=MatchLowercase}
  \newcommand{\euro}{€}
\fi
% use microtype if available
\IfFileExists{microtype.sty}{\usepackage{microtype}}{}
\ifxetex
  \usepackage[setpagesize=false, % page size defined by xetex
              unicode=false, % unicode breaks when used with xetex
              xetex]{hyperref}
\else
  \usepackage[unicode=true]{hyperref}
\fi
\hypersetup{breaklinks=true,
            bookmarks=true,
            pdfauthor={},
            pdftitle={},
            colorlinks=true,
            citecolor=blue,
            urlcolor=blue,
            linkcolor=magenta,
            pdfborder={0 0 0}}
\urlstyle{same}  % don't use monospace font for urls
\setlength{\parindent}{0pt}
\setlength{\parskip}{6pt plus 2pt minus 1pt}
\setlength{\emergencystretch}{3em}  % prevent overfull lines
\setcounter{secnumdepth}{0}

\author{}
\date{}

\begin{document}

\section{\textbf{1 Introduction}}\label{introduction}

\subsection{\textbf{1.1 Overview}}\label{overview}

The objective of this document is to deliver, demonstrate and document
the first implementation of the C-lyrics software product to the
customer. It includes detailed description of the code base management
policies, mapping to the project management plan, and mappings to the
design document for verification purposes. A startup guide to the
deliverable software is also included to guide the customer

\subsection{\textbf{1.2. Scope}}\label{scope}

The intended audience of this document however is for the client and the
future maintainers of the code base who will presumably be hired by the
client. Ideally, the document should clarify the implementation
techniques by making them predictable given that they follow the prior
documents on design and management and this is the guiding principle
with which the document is drafted. Consequently, some sections will
include designs and descriptions taken from these prior documents on
design and management.

\section{\textbf{1.3 Definitions and
Acronyms}}\label{definitions-and-acronyms}

@Please fill in from prior documents, copy pasting not needed as the
text in correct format is in Sebastien's computer@

\section{\textbf{2 Verification}}\label{verification}

\section{\textbf{2.1 System Architecture}}\label{system-architecture}

The system architecture remains mostly unchanged from that which is
described in section 2 of the C-lyrics design document. The frontend and
backend are split into different repositories in the Github code base
much like their division as separate components in the system
architecture diagram in Figure 1 of the design document. The only
exception is that LyricFind will not be used as an external API because
we were unable to acquire an API key to use the service. This fact
simplifies some the data flow as well as UML class and component designs
as described in sections 2.2 and 2.3 of this document.

\section{\textbf{2.2 Data Flow}}\label{data-flow}

Since LyricFind will not be used as an external API, our original data
flow diagram, laid out in Figure 2 of the design document, is different
than what we actually built. This issue is detailed in Mark Krant's
email on February 28, which can be referenced in full in the appendix.
LyricFind was thought to be a quick and easy API call to find all the
lyrics for songs of a specified artist, however, it is not a free or
reliable service and we did not have the budget to pay for access to its
functionality. The solution we came up with is to use a web scraper to
get lyrics from LyricsFreak.com. The drawback of this solution is that
it takes many seconds to load lyrics depending on the number of songs
the queried artist has. The data flow diagram in Figure 2 of the design
document will be slightly altered due to the use of the web scraper
rather than LyricFind. The EchoNest API will still process requests from
the front end as before, but will only retrieve the auto complete
suggestions. When an artist name is submitted the request will be sent
to the web scraper php file. The php code will then go to each website
with the lyrics and retrieve the text, returning the data as a json
object.

\section{\textbf{2.3 System Design}}\label{system-design}

\subsection{\textbf{2.3.1 Server Side Class
Diagram}}\label{server-side-class-diagram}

During implementation, the class diagram for the server side php code
had to be altered for maximum efficiency. The Echonest\_Client class in
Figure 3 of the design document was kept exactly the same, while the
EchoNestConnection class was given more functionality. Since the
LyricFind aspect of the EchoNest API was not feasible, the Lyrics class
was removed and the Autocomplete class was absorbed into
EchoNestConnection. A new class called WebScraper was created to handle
the web scraping functionality and replaced Lyrics. Now, all of the
lyrics are returned using WebScraper, while autocomplete requests go
through EchoNestConnection.

\subsection{\textbf{2.3.2 Client Side Class
Diagram}}\label{client-side-class-diagram}

The client side class diagram in Figure 4 of our design document maps
out the relationships between the main components of the application.
The MVC pattern was used as a guide to layout the client side. This gave
us a better foundation of how the code should be implemented to make the
process faster and more compliant with how we planned to create the
C-Lyrics system.

\subsection{\textbf{2.3.3 UML Component
Diagram}}\label{uml-component-diagram}

The components diagram, stated in Figure 5 of our design document, shows
the basic functionality of how the C-Lyrics system works, starting with
the user and how they will interact with the interfaces and the
functions on that page. These diagrams also show how specific functions
on these interfaces interact with certain APIs and how all functionality
will work with the server. This diagram allows us to map out which
functionality needed to be created for the implementation of the
C-Lyrics system.

\subsection{\textbf{2.3.4 UML Use Case
Diagram}}\label{uml-use-case-diagram}

The use case diagram in Figure 6 of our design document maps almost
perfectly to the functionality of our implementation. All cases,
``Submit Search,'' ``Share WC to FB,'' ``Add to Cloud,'' ``Select Word
from WC,'' ``Select Song,'' ``Go to Home Page,'' and ``Go to Lyrics
Page,'' allow the user to interact with our system as expected.

\subsection{\textbf{2.3.5 UML State Machine
Diagram}}\label{uml-state-machine-diagram}

The states of our implemented system correspond to the states outlined
in Figure 7 of the design document. While our code does not use the same
boolean variables to traverse across states, the functionality and
underlying requirements to switch states remain the same.

\subsection{\textbf{2.3.6 UML Sequence
Diagram}}\label{uml-sequence-diagram}

The sequence of events for our system from Figure 8 of our design
document remains unchanged. There is one API call when the user submits
a search query and several interactions between the user and the server.
The sequence diagram will look a little different for each use case, but
the general flow of events remains the same.

\section{\textbf{3 Process Documentation}}\label{process-documentation}

\subsection{\textbf{3.1 Overview of
Processes}}\label{overview-of-processes}

Based on section 2.2.2, of the PMP, headlined ``Staff and Personnel
Plan'', each member of the development team was originally assigned
tasks milestones based on their ``prior knowledge'' with the
technologies being used, such as PHP for example. However, given certain
limitations and incidents which will be described in the following
sections, significant but not detrimental deviations from the processes
described in the PMP were undertaken. These deviations were naturally
taken in order to assure the timely and expected completion of the
software implementation, and the completion of milestones as outlined in
the PMP.

\subsection{\textbf{3.2 Code Base
Management}}\label{code-base-management}

The whole code of the project is hosted and managed by the git program
as mentioned in section 5.2 of the PMP. Specifically, the project has
been divided into three main repositories; the frontend, the backend and
a third repository for general purpose. All three of the repositories
are hosted on GitHub.com. Milestones for the frontend and backend are
kept within each respective repository. Note again that some deviations
have occured as mentioned above, but within the context of this section,
the deviations in questions are in the usage of milestones and issue
tracking within the Github.com service. Access to the code base will be
provided to the client either upon request (during development phase) or
upon completion and verification of the final product

\subsection{\textbf{3.3 Original Milestone
Assignments}}\label{original-milestone-assignments}

Below is an excerpt from the PMP with complete descriptions of
milestones D-F, these lettering naming conventions will from now on be
used to refer to the respective milestones. D-F concern the
implementation phase of C-lyrics. A-C concern the design phase which is
as of this date completed, and milestones G-I concern the testing and
final delivery phase which as of the date of this publication is due on
March 11, 2015.

\begin{itemize}
\itemsep1pt\parskip0pt\parsep0pt
\item
  Milestones D-F

  \begin{itemize}
  \itemsep1pt\parskip0pt\parsep0pt
  \item
    D: Implementation of the Home Page

    \begin{itemize}
    \itemsep1pt\parskip0pt\parsep0pt
    \item
      D.1: Search bar with autocomplete functionality when typing in an
      artist's name
    \item
      D.2: WC generation with words that can be selected to take the
      user to the Songs Page
    \item
      D.3: Share button to upload the WC to Facebook
    \item
      D.4: Add to Cloud button to create a new WC based off of words
      commonly used by both of the specified artists
    \end{itemize}
  \item
    E: Implementation of the Songs Page

    \begin{itemize}
    \itemsep1pt\parskip0pt\parsep0pt
    \item
      E.1: List of songs sorted by how frequently the selected word is
      used in each song
    \item
      E.2: Song titles in list able to be selected, taking the user to
      the Lyrics page
    \item
      E.3: Back to Home button takes the user back to the Home Page with
      the WC still displayed and the artist's name still in the Search
      Bar
    \end{itemize}
  \item
    F: Implementation of the Lyrics Page

    \begin{itemize}
    \itemsep1pt\parskip0pt\parsep0pt
    \item
      F.1: Lyrics displayed on page with the selected word highlighted
      every time it appears in the song
    \item
      F.2: Back to Songs button takes the user back to the Songs Page
      with the same list of songs still displayed in the same order
    \item
      F.3: Back to Home button takes the user back to the Home Page with
      the WC still displayed and the artist's name still in the Search
      Bar
    \end{itemize}
  \end{itemize}
\end{itemize}

Deliverables with an asterisk (*) by the due date indicate that there
are risk factors that may alter the completion date. These risk factors
include, but are not limited to, changing requirements for each
deliverable and the deliverables taking either more or less time than
expected. Due dates for each milestone are listed in the schedule in
section 4.1 of the PMP, risk factors can be found in section 6 of the
PMP titled ``Risk Management Plan''. Also note that these dates are
subject to change by the client and are only tentative (dates are from
project schedule given by client, where it is noted that they are
subject to change).

To re-iterate the information from the above chart taken from the PMP,
and to conform the the following sub sections presentation of changes to
the assigned milestones, refer to the below list of milestones and
assignees: \emph{D: Justine, Milad }E: Kelsey, Jeff *F: Mark, Sebastien

\subsection{\textbf{3.3 Current Milestone Assignments and
Deviations}}\label{current-milestone-assignments-and-deviations}

\subsubsection{\textbf{3.3.1 Milestone
Deviations}}\label{milestone-deviations}

Changes made to milestone assignees: * D: Sebastien/Justine * E:
Sebastien/Kelsey * F: Sebastien

The above changes in assignees were a direct consequence of changes to a
team member's expected contribution to the implementation. Milad
Gueramian did not have access to the required development tools due to
the loss of visual output from his Hewlett Packard laptop computer which
was expected for use in development of software covered in milestone D.
Consequently, it became necessary to send the machine in for repairs and
this team member's role was repurposed, which is explained later in this
section.

Added milestones not specifically accounted for in the PMP: * (New)
Backend: Mark * (New) Documentation: Milad, Jeff

As mentioned before, the PMP states that the development team process
relies on a principle of matching member tasks based on his/her level of
prior experience. Through the evolution of our processes, the
development team, in an effort to be efficient, relies ever more on this
principle. While the time allocation and estimation of milestones D-F
were correct at the conception of the PMP, articulation of specific
details were not. Therefore, new milestones have since been added to
clearly separate and assign tasks to the appropriate team member based
on prior experience. Furthermore, the task for the creation of this
document was not specifically taken into account in the PMP.
Consequently, Jeff Kang and Milad Gueramian-due to the latter's
inability to contribute to milestone D-were reassigned to the new
Documentation milestone.

\subsection{3.3.2 Deviations in Project Monitoring
Plan}\label{deviations-in-project-monitoring-plan}

\textbf{Issue Tracking on Github} In section 5.4 of the PMP, it is
stated that the Github.com issue tracking system will be used to monitor
milestones. Indeed, this is a fitting management and documentation tool
since the future maintainers of the code base will want to have access
to the development history of the software. The milestones and assignees
were not added in this system until very recently and all at once, which
is not ideal because there is no smooth timeline documenting events as
they occurred. However, section 5.4 also indicates that Google Docs will
be used as a means of tracking issues. This is still true and has helped
the development team to accurately keep track of issues and milestones
and to retroactively add them in the Github issue tracker and progress
reporting system. Furthermore, other communication methods described in
earlier documents such as a group text messaging thread and a Google
email messaging thread have been useful in keeping track of progress.
Therefore, the effects of this deviation were not detrimental in making
the planned progress.

\subsection{\textbf{3.3.3 Deviations Caused By Change in
Timeline}}\label{deviations-caused-by-change-in-timeline}

The client pushed up the due date for the product implementation
deliverable from Wednesday March 4, 2015 to Monday March 2, 2015. This
put considerable constraints on the development team in completing the
milestones set for the original due date at the time of the new, March 2
due date. Consequently, the development team decided to deviate plans
for adding some added advantages and assurances to the client and
software. Sphinx as a documentation management tool, outlined in section
5.1 of the PMP, will not be used. The impact of this deviation is that
the documentation provided by Sphinx form the source code comments will
not be created and therefor not available on the code base in Github.

Testing efforts were also affected by the constrained timeline. The
tests were not as inclusive as was previously planned. Because of the
earlier due date, quality assurance is hampered because we did not have
the necessary time to achieve full coverage of test cases. As stated in
section 5.3 of the PMP on software quality assurance, we wanted to adopt
the Test Driven Development Methodology because, ``this aims to provide
an additional measure for the advancement of the project as well as an
assurance on its quality,'' the shorter deadline did not allow for this
to happen. As a result, JSCoverage was not used because our tests were
not as strong as we expected.

\subsection{\textbf{3.3.4 Quality Assurance}}\label{quality-assurance}

Travis CI was used in conjunction with Github allowing continuous
integration along the process of the limited testing which we did
perform. These limited tests were very simple however. PHP\_Unit was
also used as a code coverage tool as reported in section 5.3 (Quality
Assurance Plan section) of the PMP. Refer to the Appendix 6.1.2 for
updates on milestone from the Github issue tracker that was used, as
stated in the management plan, as a means to track progress.

Our custom progress metric and measurement system as described in the
PMP was deployed with success. Each group member of the development team
rated the progress of other members on assigned milestones and the
totals were averaged as described in the PMP. These are internal
controls that were recorded on the whiteboard of the room used for
weekly (sometimes bi-weekly) group meetings. Progress was effectively
tracked so the development team was aware of overall performance.

Additionally, progress of the individuals who coded the program was
measured by assigning primary and secondary coders. The primary was the
individual assigned to implement a function, and the secondary was the
coder who double checked their work. This was a measure of how
efficiency of implementation was accounted for to make sure the C-Lyrics
system would work.

\section{\textbf{4 Risk Monitoring Plan
Adherence}}\label{risk-monitoring-plan-adherence}

The risks identified in section 6.1 of the PMP helped prepare the
development team for the event that a team member becomes incapable of
working. A team member did not have access to his laptop and development
software and environments due to a broken screen. This caused the team
to change and re-assign some milestones as noted in the deviations
sections above. The changes were tolerable because this risk had been
accounted for in Contingency Plan \#4 of the PMP. Contingency plan \#4
also helped in creating some new milestones and re-assigning others
because the earlier estimations in the PMP were inaccurate in that the
work was not spread out efficiently and the documentation process for
the project was not taken into account as separate milestones. These
``new and re-assigned milestones'' are described in section 3 of this
document outlining deviations from milestones and listing current
milestones.

Furthermore, we accounted for the risk of not having a complete
implementation by the due date. We resorted to Contingency Plan \#6 in
our first submission of the product since some parts were not
implemented. Refer to Appendix 6.1.2 to see the functionality that is
missing and how this issue was tracked and maintained by the development
team.

\section{5 Startup Instructions}\label{startup-instructions}

\begin{itemize}
\itemsep1pt\parskip0pt\parsep0pt
\item
  Open the virtual machine image provided in the submission.
\item
  Execute the command ``sudo /opt/lampp/lampp start'' (without the
  quotation marks) and open your favorite browser to the following
  address: http://localhost/dist
\item
  Hopefully the website should appear. Please note that the application,
  especially the communication to the server, might feel a bit sluggish.
  This is probably due to latency problems between the machine and the
  API, however such problems should disappear as the application is
  deployed to a production server.
\end{itemize}

\section{\textbf{6. Appendices}}\label{appendices}

\subsection{\textbf{6.1 Issue Tracking And Progress
Reporting}}\label{issue-tracking-and-progress-reporting}

\subsubsection{6.1.1 Gmail Communications}\label{gmail-communications}

From: Sebastien Arnold, February 28, 2015

So, I just pushed some code on the organization repo.
(https://github.com/C-Lyrics/frontend) It is not fully working yet, and
there are still a few things to implement, but we are in good shape.
From what I understood, Mark has been making some progress as well on
the PHP, but we have troubles getting the actual lyrics for a given
song, as well as performance issues.

Performance should not be that much of a problem, as we can always speed
things up. Regarding the songs issue, I found two unofficial APIs for
RapGenius:
http://blog.edforson.me/introducing-genius-api-rapgenius-api-as-a-service/
and more complete, but more difficult to understand:
http://api.genius.com/search?q=the+recipe

It would be very good if we could have the backend online as well, on
the repo of the organization. So that we can have a look at how we want
to integrate both parts of the application. You should all be invited to
join, if you were not shoot me an email.

Also, someone should begin to take of the document we have to submit for
the implementation. Milad, I suggest you do it, that will be easier for
you than to code stuff, as you don't have your usual tools for coding.

From Mark Krant, February 28, 2015 Just committed the back end. It
works, but to an extent. Since I could not find a free lyrics API, I
used a webscraper framework to go to a page for each song by an artist
and get the lyrics from there. Only problem is that it takes way too
long, and Sebastien said it can be done using asynchronous calls, but I
still need to look into that more because I have no idea how to do that.
A band with like 15-20 songs might take 10-15 seconds to load, but 150+
songs will cause it to time-out (it times out after 120 seconds). If
anyone has suggestions / knows how to do it feel free to edit the code.
It is under /backend/templates/getSongs.php

Also i did not do error checking for the back end yet. so if you try to
make a word cloud with an artist name ``sdfgh'' then it will blow up.
it's a quick fix so ill do it soon

\subsubsection{\textbf{6.1.2 Milestone
Tracking}}\label{milestone-tracking}

Below are the two issues relating to unimplemented functions from the
first product submission.

\end{document}
