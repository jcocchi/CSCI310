\documentclass[]{article}
\usepackage[T1]{fontenc}
\usepackage{lmodern}
\usepackage{amssymb,amsmath}
\usepackage{ifxetex,ifluatex}
\usepackage{fixltx2e} % provides \textsubscript
% use upquote if available, for straight quotes in verbatim environments
\IfFileExists{upquote.sty}{\usepackage{upquote}}{}
\ifnum 0\ifxetex 1\fi\ifluatex 1\fi=0 % if pdftex
  \usepackage[utf8]{inputenc}
\else % if luatex or xelatex
  \ifxetex
    \usepackage{mathspec}
    \usepackage{xltxtra,xunicode}
  \else
    \usepackage{fontspec}
  \fi
  \defaultfontfeatures{Mapping=tex-text,Scale=MatchLowercase}
  \newcommand{\euro}{€}
\fi
% use microtype if available
\IfFileExists{microtype.sty}{\usepackage{microtype}}{}
\ifxetex
  \usepackage[setpagesize=false, % page size defined by xetex
              unicode=false, % unicode breaks when used with xetex
              xetex]{hyperref}
\else
  \usepackage[unicode=true]{hyperref}
\fi
\hypersetup{breaklinks=true,
            bookmarks=true,
            pdfauthor={},
            pdftitle={},
            colorlinks=true,
            citecolor=blue,
            urlcolor=blue,
            linkcolor=magenta,
            pdfborder={0 0 0}}
\urlstyle{same}  % don't use monospace font for urls
\setlength{\parindent}{0pt}
\setlength{\parskip}{6pt plus 2pt minus 1pt}
\setlength{\emergencystretch}{3em}  % prevent overfull lines
\setcounter{secnumdepth}{0}

\author{}
\date{}

\begin{document}

\title{C-Lyrics - Software Project Management Plan}

\section{1. Introduction}\label{introduction}

\subsection{1.1 Project Overview}\label{project-overview}

The primary objective of this project is to better understand the
software development lifecycle. For this specific project the objective
is to create a software system that allows a user to search for an
artist of their choice and generate a word cloud containing the most
popular used words from their lyrics database. This program will be
called C-lyrics, and is tentatively set to be delivered and readily
available for public use by March 11, 2015.

There are three major milestones that will be completed and have
subsections under each. These milestones consist of the completion of
all requirements set for each interface page, such as the home page,
songs page, and lyrics page. More details on these milestones can be
read in section 4.2. The master schedule will include five phases of
product development. These phases include designing the user interface,
implementing each function within the interfaces, testing these
interfaces, adding finishing touches, checking with the client, and
submitting. A more detailed Gantt chart will be located in section 4.1
and a more detailed description of costs can be found in section 3.

By completion, contributing members of this project should be able to
fully understand a product lifecycle as well as carrying out a complete
deliverable to all stakeholders.

\subsection{1.2 Project Deliverables}\label{project-deliverables}

The contributing authors to this documentation are the members of the
development team. They will create a fully functioning software program
that will allow anyone to specify any artist and generate a word cloud
comprised of that artist's most frequently used words. This software
program will also have the functionality to list the specified artist's
songs containing any word from the generated word cloud, and will allow
the user to select one of these listed songs to display its lyrics. This
program will be delivered as a product called C-lyrics. The client shall
receive all documentation associated with the product as well as
complete access and control of the product itself on March 11, 2015 at
THH 208 and on Blackboard. Only a single copy of each deliverable shall
be provided. These deliverables shall include software deliverables and
document deliverables. The software deliverables will include a working
software that will satisfy all set requirements and specifications made
at the beginning the project, and delivered online through Blackboard.
Document deliverables will include this current document as well as past
and previous documents to completely satisfy the clients' request.

\subsection{1.3 Evolution of the SPMP}\label{evolution-of-the-spmp}

The SPMP for C-lyrics will be available under a version control tool,
GIT, monitored by all contributing members, so any changes will be made
to the plan itself. The clients will be updated when significant changes
have been made to the prototype. The client and development team will be
able to access the updated document.

\subsection{1.4 Reference Materials}\label{reference-materials}

{[}1{]} IEEE. IEEE Std 830-1998 IEEE Recommended Practice for Software
Requirements Specifications. IEEE Computer Society, 1998.

{[}2{]} ``word cloud''.
\href{http://www.oxforddictionaries.com/us/definition/american_english/word-cloud}{Oxforddictionaries.com}
(January 31, 2015)

{[}3{]} EchoNest API
\href{http://developer.echonest.com/docs/v4/index.html\#overview}{documentation.}
(January 29, 2015)

{[}4{]} Van Vliet, Hans. Software Engineering: Principles and Practice.
3rd ed. Indianapolis, Indiana: Wiley, 2008. 170-71.

\subsection{1.5 Definitions and
Acronyms}\label{definitions-and-acronyms}

Term Definition AJAX Asynchronous JavaScript And XML. Technology
allowing the transfer of data from between the front- and back-end
without reloading the web page. API (EchoNest) API will refer to the
EchoNest API. EchoNest is a free API that allows developers to retrieve
lyrics and artist information in web pages and other programs.
Autocomplete Autocomplete refers to the functionality addition to the
Search Bar, allowing users to enter minimal characters and choose
artists that are most similar to the string and display a picture of
those artists next to their name. Autocomplete Delay A feature designed
for the search bar when a user is typing. The delay refers to the
suspending action while the user is typing, making the request to the
server for autocomplete. Backend References the PHP backend page Back to
home button A button redirecting the user to the homepage. Back to songs
button A button redirecting the user to the songs list page. Commonly
Used Web Browser Browsers such as Firefox, Safari, Chrome, Explorer, and
Quora which come on mobile phones, tablets and personal computers.
Customer/Client Dr.~William G. Halfond and Sonal Mahajan GitHub A web
service that provides software version control tools. www.github.com
Stakeholders The client and the development team LOC acronym: for Lines
of Code KSLOC a metric that stands for: 1,000(K) Source Lines of Code
Desktop Platform A screen whose width exceeds 560px Development Team All
of the individuals whose names appear on the cover of this document.
These persons have collectively put this document together and will
collectively implement the software product described in subsequent
sections. Facebook Online social network service where the generated
word cloud image may be shared amongst users. Google Doc An online
service provided by Google Inc. where an editable document can be
accessed and change simultaneously by the members who have been given
access to the document. In the case of the development team, google doc
is the shared resource which contains the source of this SRS document.
Home Page The first page of the website visited by the user. It contains
the Word Cloud as well as the Search Bar. Lyrics Page The third page of
the website, it contains the lyrics for one song, which is chosen by the
user on the Songs Page. It will have two Navigation Buttons that can
take the user to either the Home Page or back to the Songs Page. Mobile
Platform A screen whose width is less than or equal to 560px Navigation
Buttons Refers to any button that takes the user to previously visited
pages of the website. Software Project Management Plan (SPMP) Refers to
this document. Prototype A small prototype of the software including the
barebones of the graphical display. Used during the second meeting with
the client, screenshots available in the appendices. Search Bar The
initial search bar on the first page of the website. Here, users can
type in artist or band names to generate a word cloud. Share Button The
standard, embeddable Facebook share button. Software or Product The
application software delivered from the supplier to the customer. Song
List This will be the culmination of all songs found that contain the
search word indicated by the user. Songs Page The second page of the
website. It contains the Song List as well as a Navigation Button back
to the Home Page. The user navigates to the Songs page by clicking on a
word in the Word Cloud on the Home page. Submit Button The button
adjacent to the Search Bar. When the user enters an artist name into the
Search Bar and is ready to generate the Word Cloud, he or she must click
on the Submit Button to begin the process. Supplier The team developing
the product for the customer. System The set of machines running the
software making it accessible to the user. User A person who interacts
with C-lyrics software Word Cloud (WC) A word cloud (otherwise known as
a tag cloud) is, according to the Oxford Dictionary, an image composed
of words used in a particular text or subject, in which the size of each
word indicates its frequency or importance {[}2{]}.

\section{2. Process and Organization}\label{process-and-organization}

\subsection{2.1 Software Lifecycle
Process}\label{software-lifecycle-process}

The software lifecycle process for this project will be based on the
Waterfall Model. In this model, each phase of software development needs
to be completed before attempting to work on the next phase. Testing is
the last phase of the process which takes place after the software has
been implemented.

\subsection{2.2 Team Organization}\label{team-organization}

The organizational structure of our team is flat with no specified
leader or hierarchy. Our team is structured by collaboration across all
members resulting in group decisions on who will work on each part of
the overall project. All internal deadlines and deliverables are set as
a team and communicated to everyone during team meetings and emails.
Each member of the team is equally responsible for every finished
deliverable as there is no team lead for any specific component.

\subsubsection{2.2.1 Current Group Process}\label{current-group-process}

The current group process in place for the development team is to meet
as an entire group whenever it is deemed necessary by a general
consensus. The consensus is gathered by means of group text messaging.
At each meeting, the development team decides the tasks to be completed,
progress made as of the time of the meeting and delegates future tasks
to each member for completion. This process has proved itself highly
effective as of the date of this publication and it is closely estimated
to continue through the entire life cycle of the software development
process for C-lyrics.

\subsubsection{2.2.2 Staff and Personnel
Plan}\label{staff-and-personnel-plan}

The C-Lyrics team will consist of the members of the development team.
Each member is proficient in object oriented design and concepts, and
has some familiarity with web development. Knowledge of git and other
version control methods is a must. The duration of need will be until
the last milestone and deliverable is reached. Each member will be
assigned tasks based on his or her prior knowledge. Since there will be
no training, members will have to research on the job if they are
insufficiently prepared. Personnel will not be phased out and retention
will not be a factor.

\subsubsection{2.2.3 Staff Allocation}\label{staff-allocation}

SEE PICTURES

\subsubsection{2.2.4 Team Member
Description}\label{team-member-description}

This section provides a brief description of the members of the
development team and their relative experiences. Each member is a
student at the University of Southern California and is affiliated with
the Viterbi School of Engineering Computer Science Department. For
additional information on each member, refer to appendix for resumes.

\section{3. Cost Estimation}\label{cost-estimation}

\subsection{3.1 Cost Influence}\label{cost-influence}

The cost influencing factors that have been considered follow the
standard COCOMO II model for estimating project effort. These include
the four main categories of product factors, platform factors, personnel
factors, and project factors. The detailed list of cost drivers and the
associated estimations can be found in section 3.3 of this document.

\subsection{3.2 Cost Schedule
Relationship}\label{cost-schedule-relationship}

Cost and schedule are related based on the fact that this particular
project contains no defined wages in terms of fiscal costs. Therefore,
the only real ``cost'' is the effort and time that will be put into
creating the C-lyrics project. In other words, the schedule is perfectly
correlated with the cost of the project.

\subsection{3.3 Development Costs}\label{development-costs}

Our development costs have been estimated using the COCOMO II Model. The
following results are estimates of the cost drivers that are applied to
the model {[}4{]}.

----check table---- Product Factors Reliability required: very low, 0.75
Database size: low, 0.93

Product complexity: low, 0.88 Required reusability: nominal, 1.00
Documentation needs: very high, 1.13

The project requires very low reliability, since it would only pose a
slight inconvenience if the C-lyrics platform encountered problems and
was not available. We will most likely not need a database since the
lyrics generation will be based on an online API, therefore the database
size is listed as low. The low product complexity was selected based on
an average of the nominal control operations level and the very low
computational operations level. The reuse of components in the product
is nominal, since components only need to be reused within the project.
The documentation relative to the complexity of our project is very
high.

---check table--- Platform Factors Execution-time constraints: nominal,
1.0 Main storage constraints: nominal, 1.0 Platform volatility: low,
0.87

The execution-time constraints are nominal since we will require less
than 50\% of available execution time. The main storage constraints are
nominal since we will require less than 50\% of available system
storage. Platform volatility is low since the hardware and software
required to run our program will not need frequent updates.

---check table--- Personnel Factors Analyst capability: very low, 1.50
Programmer capability: low, 1.16 Application experience: very low, 1.22
Platform experience: low, 1.10 Language and tool experience: low, 1.12
Personnel continuity: very high, 0.84

Analyst capability is very low since the group's ability to create
detailed, high-level design for software projects is minimal. Programmer
capability is low because of the lack of significant industry
programming experience of the team. Language and tool experience is low
since most of the team's programming experience centers around one or
two functional languages and little experience with software tools.
Personnel continuity is very high since all members of the team will
continue to be working together over the course of the project.

---check table--- Project Factors Use of software tools: low, 1.12
Multi-site development: very high, 0.84 Required development schedule:
very high, 1.0

The use of software tools is designated as low because we will not be
using highly developed tools that are structured for the purposes of our
project. The multi-site development rating is very high, since the group
has consistent communication both in person and through electronic
channels. The required development schedule is listed as very high since
the implementation phase of our project exceeds 160\% of the expected
time to complete the project.

The effort in terms of person-months is calculated as the KSLOC
multiplied by the product of all cost drivers according to COCOMO II.
With an estimated KSLOC of 1, the number of person-months that this
calculation returns is 1.247.

\subsection{3.4 External Costs}\label{external-costs}

Some of the external costs that are not directly factored into the
project costs include: purchasing the class textbook for research,
programming resources, and transportation costs for group incured by
individual members for meetings.

\section{4. Schedule, Milestones, and
Deliverables}\label{schedule-milestones-and-deliverables}

\subsection{4.1 Schedule}\label{schedule}

SEE PICTURE

\subsection{4.1.1 Activity Network}\label{activity-network}

(See section 4.2 for the specific task encompassed in each milestone.)

\subsection{4.2 Project Milestones and
Deliverables}\label{project-milestones-and-deliverables}

The process model for our project is the Waterfall Model, meaning we
will finish each big software development phase before we start the
next. We will finish requirements engineering before starting the
project design, which will be finished before we start implementation,
which will be finished before we start testing, which will be finished
before we arrive at the final phase of project, maintenance.

Our project initiation was being assigned this project by Dr.~Halfond.
The first step in this project was beginning to work on the Software
Requirements Specification. The subsequent milestones and deliverables
are listed below.

Deliverables and Milestones:

\begin{itemize}
\itemsep1pt\parskip0pt\parsep0pt
\item
  Project Management
  Plan\ldots{}\ldots{}\ldots{}\ldots{}\ldots{}\ldots{}\ldots{}\ldots{}\ldots{}\ldots{}\ldots{}\ldots{}\ldots{}\ldots{}\ldots{}\ldots{}.2/9/15
\item
  There are no specific milestones for this deliverable as it will be
  worked on by all team members and completed in full by February 9th.
\item
  Design\ldots{}\ldots{}\ldots{}\ldots{}\ldots{}\ldots{}\ldots{}\ldots{}\ldots{}\ldots{}\ldots{}\ldots{}\ldots{}\ldots{}\ldots{}\ldots{}\ldots{}\ldots{}\ldots{}\ldots{}\ldots{}\ldots{}\ldots{}\ldots{}2/18/15*
\item
  Milestones A-C (2 SPACES)

  \begin{itemize}
  \itemsep1pt\parskip0pt\parsep0pt
  \item
    A: Design of the Home Page ( 5 SPACES)

    \begin{itemize}
    \itemsep1pt\parskip0pt\parsep0pt
    \item
      A.1: Search bar (9 SPACES)
    \item
      A.2: WC generation
    \item
      A.3: Share button
    \item
      A.4: Add to Cloud
    \end{itemize}
  \item
    B: Design of the Songs Page

    \begin{itemize}
    \itemsep1pt\parskip0pt\parsep0pt
    \item
      B.1: List of songs
    \item
      B.2: Back to Home button
    \end{itemize}
  \item
    C: Design of the Lyrics Page

    \begin{itemize}
    \itemsep1pt\parskip0pt\parsep0pt
    \item
      C.1: Lyrics displayed on page
    \item
      C.2: Back to Songs button
    \item
      C.3: Back to Home button
    \end{itemize}
  \end{itemize}
\item
  Implementation\ldots{}..\ldots{}\ldots{}\ldots{}\ldots{}\ldots{}\ldots{}\ldots{}\ldots{}\ldots{}\ldots{}\ldots{}\ldots{}\ldots{}\ldots{}\ldots{}\ldots{}\ldots{}\ldots{}..3/4/15*
\item
  Milestones D-F

  \begin{itemize}
  \itemsep1pt\parskip0pt\parsep0pt
  \item
    D: Implementation of the Home Page

    \begin{itemize}
    \itemsep1pt\parskip0pt\parsep0pt
    \item
      D.1: Search bar with autocomplete functionality when typing in an
      artist's name
    \item
      D.2: WC generation with words that can be selected to take the
      user to the Songs Page
    \item
      D.3: Share button to upload the WC to Facebook
    \item
      D.4: Add to Cloud button to create a new WC based off of words
      commonly used by both of the specified artists
    \end{itemize}
  \item
    E: Implementation of the Songs Page

    \begin{itemize}
    \itemsep1pt\parskip0pt\parsep0pt
    \item
      E.1: List of songs sorted by how frequently the selected word is
      used in each song
    \item
      E.2: Song titles in list able to be selected, taking the user to
      the Lyrics page
    \item
      E.3: Back to Home button takes the user back to the Home Page with
      the WC still displayed and the artist's name still in the Search
      Bar
    \end{itemize}
  \item
    F: Implementation of the Lyrics Page

    \begin{itemize}
    \itemsep1pt\parskip0pt\parsep0pt
    \item
      F.1: Lyrics displayed on page with the selected word highlighted
      every time it appears in the song
    \item
      F.2: Back to Songs button takes the user back to the Songs Page
      with the same list of songs still displayed in the same order
    \item
      F.3: Back to Home button takes the user back to the Home Page with
      the WC still displayed and the artist's name still in the Search
      Bar
    \end{itemize}
  \end{itemize}
\item
  Testing and Final
  Delivery\ldots{}..\ldots{}\ldots{}\ldots{}\ldots{}\ldots{}\ldots{}\ldots{}\ldots{}\ldots{}\ldots{}..\ldots{}\ldots{}\ldots{}\ldots{}3/11/15*
\item
  Milestones G-J

  \begin{itemize}
  \itemsep1pt\parskip0pt\parsep0pt
  \item
    G: Testing of the Home Page

    \begin{itemize}
    \itemsep1pt\parskip0pt\parsep0pt
    \item
      G.1: Search bar with autocomplete functionality when typing in an
      artist's name
    \item
      G.2: WC generation with words that can be selected to take the
      user to the Songs Page
    \item
      G.3: Share button to upload the WC to Facebook
    \item
      G.4: Add to Cloud button to create a new WC based off of words
      commonly used by both of the specified artists
    \end{itemize}
  \item
    H: Testing of the Songs Page

    \begin{itemize}
    \itemsep1pt\parskip0pt\parsep0pt
    \item
      H.1: List of songs sorted by how frequently the selected word is
      used in each song
    \item
      H.2: Song titles in list able to be selected, taking the user to
      the Lyrics page
    \item
      H.3: Back to Home button takes the user back to the Home Page with
      the WC still displayed and the artist's name still in the Search
      Bar
    \end{itemize}
  \item
    I: Testing of the Lyrics Page

    \begin{itemize}
    \itemsep1pt\parskip0pt\parsep0pt
    \item
      I.1: Lyrics displayed on page with the selected word highlighted
      every time it appears in the song
    \item
      I.2: Back to Songs button takes the user back to the Songs Page
      with the same list of songs still displayed in the same order
    \item
      I.3: Back to Home button takes the user back to the Home Page with
      the WC still displayed and the artist's name still in the Search
      Bar
    \end{itemize}
  \item
    J: Testing of the entire product to ensure all pages work together
    as specified in the SRS
  \end{itemize}
\end{itemize}

Deliverables with an asterisk (*) by the due date indicate that there
are risk factors that may alter the completion date. These risk factors
include, but are not limited to, changing requirements for each
deliverable and the deliverables taking either more or less time than
expected. Due dates for each milestone are listed in the schedule in
section 4.1. Also note that these dates are subject to change by the
client and are only tentative (dates are from project schedule given by
client, where it is noted that they are subject to change)

The termination activity of our project is presenting the completed
product to both Dr.~Halfond and Ms.~Mahajan and ensuring that both of
them are satisfied with our work. However, if this is not accomplished
by the final project deadline, 3/11/15, then our project will be
terminated regardless of the customer's satisfaction.

\section{5. Configuration Management
Plans}\label{configuration-management-plans}

\subsection{5.1 Product Documentation
Management}\label{product-documentation-management}

Formal report formats between the client and development team will be
structured according to IEEE standards and use LaTeX as a document
preparation system for the client. The code documentation will be
generated by Sphinx, which provides an easy way to create intelligent
and beautiful documentation. In particular, it allows the documentation
to be generated in several formats (eg, HTML, LaTeX, PDF) directly from
the comments in the source code. Therefore, the documentation will
partly be integrated in the code base and will be versioned by git as
well. These processes will be used for the Software Requirements
Specifications (SRS), Software Project Management Plan (SPMP), and a
Design Document (DD).

\subsection{5.2 Code Base Management}\label{code-base-management}

The whole code of the project will hosted and managed by the git
program. Specifically, the project is going to be divided in three main
repositories; the frontend, the backend and a third repository for
general purpose. All three of the repositories will be hosted by
GitHub.com and they will belong to the C-Lyrics organisation. This will
allow the development team to better coordinate their effort while
creating the software, and takes care of most of the merges in the code
base. As stated in Section 5.3, the GitHub repositories will be coupled
with Travis CI for continuous integration.

\subsection{5.3 Software Quality Assurance
Plan}\label{software-quality-assurance-plan}

Reviews and audits will take place after every milestone is completed.
The development team will review the progress made thus far to make sure
deadlines will be met, while observing and complying with the details of
the project management plan. If there are noticeable shortcomings and
deviations in the current progress of the project from what has been
decided in the project management plan, audits to the project will have
to be made.

In addition, the software will be developed according the Test Driven
Development methodology. This aims to provide an additional measure for
the advancement of the project as well as an assurance on its quality.
By writing the basic tests first and then implementing the functions so
that they pass the tests, the goal of the development process is
slightly modified. Instead of aiming to implement every requirement and
functionality, the objective is to have all the pre-written tests
pass.To keep track of such advancements, GitHub will be coupled with
Travis CI allowing continuous integration along the process. Finally
some code coverage tools (ie, JSCoverage, PHP\_Unit) will be used in
order to keep track of the percentage of total coverage on the project.

In order to keep track of the progress, the development team will use a
custom metric. Essentially, the metric mixes the objectivity of the test
coverage as well as the personal insights of the team members. Each team
member will give a grade from 1 to 5 to each one of the other members
based on the amount of work they accomplished. Each member's score will
be averaged and the average of these averages will be the final score of
the team. This score is then combined with the amount of tests that pass
as well as the amount of code coverage. Hopefully, this combination of
both an objective metric as well as the team's self opinion on how they
are doing provides an accurate estimation of the progress on the
project. The mathematical details and some additional reasons for the
use of this metric are included in the Appendix.

\subsection{5.4 Project Monitoring Plan}\label{project-monitoring-plan}

Github's issue tracker and progress reporting system, along with Google
Docs, will be used as the primary project monitoring mechanism among
team members. Tasks will be assigned to each team member with a
description and a project milestone. Each issue completion will
correspond to a milestone set in the project management plan. If there
is a change in the milestone contents and dates, the issue tracker will
be updated in accordance and strictly follow the most up to date plan.
Once the task is completed, the team member will submit the deliverables
to Github and mark the task in the issue tracker as completed.

\section{6. Risk Management Plan}\label{risk-management-plan}

\subsection{6.1 Risk Identification}\label{risk-identification}

SEE TABLE Risk Type Possible Risks Technology The maximum number of
queries allowed on EchoNest is reached during testing. The server that
is hosting the domain goes down. The website traffic is overloaded,
causing a crash. The real-time performance of the software is
inadequate. People Staff members do not have the required technical
skills demanded by the project and must spend time to learn them. One or
more members of the staff are seriously ill or incapable of working.
Staff members who lack experience are unable to learn quick enough to
meet deadlines. Organisational One or more new staff members are added
to the development team. One or more of the current staff members quits.
A flat organization produces a lack of accountability and confusion,
resulting in delayed deliverables. The waterfall method is ineffective
due to changing customer demands and uncertainties. Metrics The metrics
are not completely objective Tools The chosen CASE tools cannot be
integrated into the software without serious restructuring. Requirements
The client and stakeholder want to alter small details of the product.
The client and stakeholder want to restructure the product entirely.
Requirements from the client are misinterpreted during implementation.
The development team implements features that looks aesthetically
pleasing, but is unwanted by the client. Estimation The software takes
longer to develop than previously anticipated. The number of people
using the software is greater than what was previously decided by the
client and stakeholder.

\subsection{6.2 Risk Analysis}\label{risk-analysis}

SEE TABLE No. Risk Probability Effects 1 The maximum number of queries
allowed on EchoNest is reached during testing. Very Low Catastrophic 2
The server that is hosting the domain crashes. Low Serious 3 The website
traffic is overloaded, causing a crash. Very Low Serious 4 Staff members
do not have the required technical skills demanded by the project and
must spend time to learn them. Moderate Tolerable 5 One or more members
of the staff are seriously ill or incapable of working. Low Serious 6
Staff members who lack experience are unable to learn quick enough to
meet deadlines. Low Serious 7 One or more new staff members are added to
the development team. Low Insignificant 8 One or more of the current
staff members quits. Low Tolerable 9 The chosen CASE tools cannot be
integrated into the software without serious restructuring. Moderate
Tolerable 10 The client and stakeholder request to alter small details
of the product. High Tolerable 11 The client and stakeholder want to
restructure the product entirely. Moderate Catastrophic 12 The software
takes longer to develop than previously anticipated. High Tolerable 13
The number of users is greater than what was previously decided by the
client and stakeholder. Moderate Tolerable 14 A flat organization
produces a lack of accountability and confusion, resulting in delayed
deliverables. Moderate Serious 15 The waterfall method is ineffective
due to changing customer demands and uncertainties. Moderate Tolerable
16 The real-time performance of the software is inadequate. Moderate
Serious 17 Requirements from the client are misinterpreted during
implementation. High Tolerable 18 The development team implements
features that looks aesthetically pleasing, but is unwanted by the
client. Moderate Tolerable 19 Objectibility of custom metric to track
progress of project is low Moderate Tolerable

\subsection{6.3 Risk Planning}\label{risk-planning}

SEE TABLE Risk No. Strategy Type Solution 1 Avoidance Make smarter test
cases that cover more areas of vulnerability in a smaller amount of API
queries. 2 Avoidance Host the software on a reliable server, or perhaps
change the hosting location based off of previous crashing problems.. 3
Avoidance Select a more robust hosting service. 4 Contingency Plan
Allocate staff members to develop parts of the project that best suits
their skills. If no one is familiar with the language or practices
required, work in teams of two to promote collaboration 5 Contingency
Plan Assign the tasks of the ill staff member to various other members,
while the ill staff member can update the new members on the current
state of the task. 6 Contingency Plan Delay the deliverable output, or
submit the incomplete deliverable and patch the mistakes later when
given more time. 7 Minimisation Update the new staff member with the
current state of the software and introduce low level tasks to help get
him or her up to speed. 8 Minimisation Allocate the work of the quitting
staff member among the remaining team members. 9 Contingency Plan Find
new tools to fill the needs of the software. 10 Contingency Plan Update
previous documents with the new information and fill out a revision
history of each document. Change the code to fit the demands and,
depending on the magnitude of the desired changes, restructuring of the
software may be necessary. 11 Contingency Plan Depending on the amount
of change desired by the client, the staff may have to create new
documents entirely and essentially start a new project. 12 Contingency
Plan Change the deliverable dates to compensate for a new estimated time
using COCOMO. 13 Minimisation To compensate for more users than
previously intended, the software may have to be altered to more
effectively handle storage, the EchoNest API licensing may have to be
purchased to allow more requests, and the server will need to be able to
handle more requests. 14 Minimisation Hold team members more accountable
for his or her own tasks. Part of this can be achieved by instilling a
productive atmosphere during meetings and other official work times. If
there continues to be problems with a flat organization, a hierarchy may
need to be introduced to inspire more efficient work. 15 Contingency
Plan The development team will have to work around the inefficiencies of
the waterfall method. Switching to Agile methodology is not an option.
16 Avoidance The development team will need to structure the code in a
way that produces results in the quickest way possible. If it is too
slow, restructuring may be required. Caching results is one way to speed
up results. 17 Avoidance If the requirements specified by the client is
different than what is implemented, the development team will have to
fix the unwanted components. This can be avoided by following the
requirements document closely, assuming the client specified everything
during the formation of the requirements document. 18 Avoidance The
development team should follow the requirements document regardless of
the opinions of the team on the specifications of components.

\subsection{6.4 Risk Monitoring}\label{risk-monitoring}

Throughout the development cycle, the team will have to monitor each
risk regularly and prepare for it depending on the estimated likelihood
that it will occur. Each member will be aware of the risk pertaining to
his or her tasks and discuss the chance that they occur at each meeting.
Many risks do not involve the actual implementation and cannot be
anticipated, therefore these risks can be ignored during team meetings.
Larger risks for a particular milestone or deliverable can be noted in
the github issue tracker along with the regular team meeting procedure.

\end{document}
